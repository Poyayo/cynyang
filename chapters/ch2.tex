\chapter{系统设计}

“SheepOS”,是我为该操作系统取的一个名字,这名字的由来也很简单,毕竟是自己做的一个OS,那也总得自
己给它取个名字,Sheep其实跟我的名字也有一点关系,在以前初中的时候刚学没几个单词,同学们就
开始取英文名玩儿,他们简单粗暴的给我取了个“Sheep Star”的名字,现在想想还挺好笑的。但
我也并不排斥Sheep这个词,而且我也很喜欢绵羊,所以索性就把我的OS叫做SheepOS了。

\section{总体设计}
\label{sec:overalldesign}

本文主要参照Linux操作系统的内核框架,尝试实现一个带有文件操作功能和命令
行接口的简单操作系统。本系统主要包含以下四个模块:1)启动模块;2)进程
模块;3)数据存储模块;4)外围功能模块。

这四个模块分别有自己负责的范围。其中,启动模块负责从上电开机到操作系统
启动的过程;进程模块负责进程的产生、调度、和管理;数据存储模块实现硬盘、
内存等存储数据的功能,它是后续的文件系统、键盘输入等功能的基础;外围功
能模块建立在前三个模块的基础之上,主要是一些针对文件的操作,包括文件的
打开、关闭、读写、删除等等。这些功能的实现都离不开前三个模块,只有先完
成了前三个模块,才能实现这些基本的功能。

\section{模块设计}

\subsection{启动模块}

启动模块主要涉及BIOS、MBR、Loader、Kernel等部分。在操作系统启动过程中,
先由BIOS来找到MBR,
然后由MBR来引导并加载Loader,再由Loader来加载Kernel。如图
\ref{fig:boot}所示,启动模块的目的可以总结为一个,那就
是加载Kernel。它的详细过程在第\ref{cha:OSboot}章中将会介绍。

\begin{figure}[H]
  \centering
  \includegraphics[width=.6\textwidth]{boot}
  \caption{与系统启动相关的各模块}
  \label{fig:boot}
\end{figure}

\subsubsection{计算机的开机过程}

引领我们走向计算机这一整个系统的神秘代
码,\texttt{0x7C00},一切都要从它开始。计算机通电开机之后,BIOS便会开始
自检,在找到可用的磁盘后,BIOS就会把它的第一个扇区加载
到\texttt{0x7C00},之后由一个512字节的主引导记录MBR\footnote{实际上只
  有446字节用于引导程序和参数,剩下的64字节用于分区表和2字节用于结束标
  记的\texttt{0x55}和\texttt{0xAA}。}从BIOS中接过系统的控制权,也就
是CPU的使用权。MBR便是从\texttt{0x7C00}处接管CPU的,刚好是512字节。之
后,MBR寻找操作系统所在的分区,我们规定用\texttt{0x80}来表示分区上有引
导程序,方便MBR从众多分区中方找到操作系统所在的分区,MBR如果找到了这个
分区,就会将CPU使用权交给这个分区上的引导程序,该引导程序通常就是内核加
载器,所以,为了让MBR能够更方便的在那么大的分区里找到内核加载器,通常会
把内核加载器的入口地址固定在分区最开始的扇区,该扇区就是操作系统引导扇
区(MBR扇区)。而在MBR扇区的前3个字节处存放了跳转指令,目的是为了
让MBR找到分区交接工作后,将处理器带入操作系统引导程序中,至此MBR就完成
了所有工作,CPU的控制权就交到了内核手里。到此为止,计算机完成了开机,此
时计算机的状态就像是在执行代码:

\begin{codeblock}
\begin{ccode}
while(1)
{
  操作系统代码();
}
\end{ccode}  
\end{codeblock}


\subsection{进程模块}

进程模块主要包括特权级的转变、用户进程的创建、还有进程调度等方面。有了该模块操作系统才能使用户
程序在一个相对安全的环境中执行,并且能够实现多任务同时进行与切换。在本操作系统中,特权级转
变采用的是中断返回的方法:当中断发生时,在中断入口函数\texttt{intr\%1entry()}中通过\texttt{push}
操作来保存当前任务的上下文数据,因此,之后也需要有相应的\texttt{pop}操作来恢复数据,这属于
\texttt{intr\%1entry()}函数的逆过程,在使用\texttt{pop}操作恢复数据后,CPU就会认为用户进程从中断返回了。
在此操作之后,用户进程将在最低的特权级3下运行,操作系统处于最高特权级0,这样就达到了
该模块的目的。当用户进程处在特权级3之后,进程的创建就是通过一个\texttt{process\_execute()}函数来创建
的,创建成功后再由时钟中断用\texttt{schedule()}从就绪队列中进行调度。详细的步骤在第
\ref{sec:course}节中将会介绍。

\subsection{内存模块}

此模块就是为了进程而存在的,主要为新产生的进程分配对应的内存空间,在进程结束后再对内存进行
回收。内存的创建并不是提前准备好的,它是在需要的时候由程序动态创建的,例如进程需要内存的时
候,会调用相关的函数,然后动态分配并且维护内存块资源,再使用完内存后,还需要把内存回收回去,
而内存的使用情况一直是由位图\footnote{位图是操作系统中常用的一种数据结构,是一个二进制数
  组,其中的每个Bit表示相应的存储快的状态,通常用0表示未使用,1表示已使用。}来进行管理的,
所以无论内存的分配或者释放,本质上其实就是在设置相关位图的相应位,也就是在读写位图。

\subsection{外围功能模块}

在有了启动、进程、内存模块之后,就可以实现一个简单的文件系统,一个简单的shell,shell的一些
简单功能,本文中主要实现的功能有:文件的创建、文件的删除、文件的打开、文件的关闭、文件的写
入、文件的读取、文件的删除这些文件操作。

而在shell中实现了\Ctrl{L}和\Ctrl{U}快捷键,分别是清屏和清除输入,还有一些简单的\texttt{ls}、\texttt{cd}、
\texttt{mkdir}、\texttt{rmdir}操作。 

%%% Local Variables:
%%% mode: latex
%%% TeX-master: "../thesis"
%%% End:
