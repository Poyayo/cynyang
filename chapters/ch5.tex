\chapter{不足与展望}
通过对Linux操作系统的学习和本次毕业设计实践,让我对计算机的工作原理、
操作系统的概念、和系统编程有了更加深刻的了解。不足之处有以下几点:
\begin{enumerate}
\item 实现的功能较少,仅有一些简单的对文件的操作。
\item 我觉得接触到的东西虽然已经说是底层的了,但并不是最底层的,像\texttt{nasm}、\texttt{gcc}、\texttt{ld}这些指令都
  是前人写好的,直接拿来用了,以后若是有时间定会研究一下由自己写的编译器来编译自己写的程序。
\item Tab自动补全的功能,我自己在日常使用中最常用的也就是这个键,这次没有实现还是挺遗憾的。
\item 网络相关的知识,虽然现在感觉离这个层面还很远,但感觉以后若是涉及到这方面的问题一定也
  会很有趣。
\end{enumerate}

基本上就是以上这些地方感觉要是有时间的话还是能够实现的,所以就带有一丝丝遗憾,由于要准备考
试以及一些其他原因,对这次毕业设计的时间其实很紧张,然后操作系统里的很多东西都是很底层的,
像汇编、和一些基础的C语言\cite{HD2008}学习起来就需要更多的时间去了解相关的知识,最后感觉也只是了解到了
一点皮毛,说实话也就是在别人铺好的路上走了一遍,用着别人提前写好的一些框架。

其实在最初定下要研究这个题目的时候,目标是想将显示PDF这些功能也加进去,最后能在我自己制作
的SheepOS进行论文答辩,那肯定是件非常有意义的事,但现实总比理想要残酷的多,感觉还差的很远
,以后还需要继续努力,加油。

%%% Local Variables:
%%% mode: latex
%%% TeX-master: "../thesis"
%%% End:
